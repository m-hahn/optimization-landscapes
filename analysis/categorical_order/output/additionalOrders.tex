Afrikaans & SVO & \citet{gell-mann-origin-2011}\\
Archaic Greek & No dominant order & \citet{gell-mann-origin-2011} entry for `Classical Greek'\\
Classical Chinese & SVO &  \citet{pulleyblank1995outline}, unambiguously confirmed by UD data.\\
Classical Greek & No dominant order & \citet{gell-mann-origin-2011} entry for `Classical Greek'. Also labeled like this in \citet{wals-81}, Figure 1\\
Faroese & SVO & \citet{gell-mann-origin-2011}\\
Galician & SVO & \citet{gell-mann-origin-2011}\\
Gothic & SOV & \citet{gell-mann-origin-2011}\\
Kazakh & SOV & \citet{gell-mann-origin-2011}\\
Koine Greek & No dominant order & \citet{gell-mann-origin-2011} entry for `Classical Greek'\\
Latin & SOV & \citet{gell-mann-origin-2011}\\
Maltese & SVO & \citet{gell-mann-origin-2011}\\
Naija & SVO & Based on UD data\\
Old Church Slavonic & SVO & \citet{gell-mann-origin-2011}\\
 Old English & SVO & VO was dominant, compared to OV, by the 9th century \citep{west1973some}. Also \citet{wals-81} Figure 1 judges as SVO.\\
Old French & SVO & OV order in finite clauses lost by 13th century (\citet{zaring2010changing}, citing \citet{marchello-nizia1995l})\\
Old Russian & SVO & Old Russian has word order freedom similar to Old Church Slavonic, which is labeled SVO. The alternative label would be `No dominant order'.\\
Sanskrit & SOV & \citet{gell-mann-origin-2011}\\
Slovak & SVO & \citet{gell-mann-origin-2011}\\
